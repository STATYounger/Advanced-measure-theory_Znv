\documentclass{report}

\input{preamble}
\input{macros}
\input{letterfonts}

\title{\Huge{Advanced Measure Theory}}
\author{Based by Prof.Dr.Dümbgen Lutz\\ZnV\\
\textit{The life is a function of the value you bring to others.}}
\date{\today}

\begin{document}

\maketitle
\newpage% or \cleardoublepage
% \pdfbookmark[<level>]{<title>}{<dest>}
\pdfbookmark[section]{\contentsname}{toc}
\tableofcontents
\pagebreak

\chapter{Signed Measures}
\textit{Start the coming trip from \textcolor{blue}{finite signed measures}.}\\
Assume $(\Omega,\SA)$ be a measurable space, where $\Omega$ is a nonempty set equipped with a $\sigma-$algebra $\SA$ over $\Omega$.
\section{The Hahn-Jordan Decomposition}
\dfn{Finite signed measure}{
A function $\nu:\SA\to\mathbb{R}$ is called a finite signed measure on $(\Omega,\SA)$, if
\begin{enumerate}
    \item $\nu(\emptyset)=0$,
    \item $\nu$ is $\sigma-$additive.
\end{enumerate}
}
\nt{$\sigma-$additive: for arbitrary disjoint sets $A_1,A_2,A_3,\cdots\in\SA$,
\begin{align*}
    \nu(\bigcup_{n=1}^{\infty}A_n)=\sum_{n=1}^{\infty}\nu(A_n).
\end{align*}}
\qs{}{What is the difference to a finite measure?}
$\nu(A)$ may be negative for some sets $A\in\SA$.
\qs{}{Why has the series $\sum_{n=1}^{\infty}\nu(A_n)$ converge absolutely?}
Because the set $\bigcup_{n=1}A_n$ is invariant w.r.t.reorderings of the sets $A_1,A_2,\cdots$.

\ex{Finite signed measure $1$}{
\begin{align}
    \nu:=Q-P
\end{align}with finite measures $P,Q$ on $(\Omega,\SA)$.
}
\begin{myproof}
To prove that \(\nu := Q - P\) with finite measures \(P\) and \(Q\) on \((\Omega, \mathcal{A})\) is a finite signed measure, we need to verify that \(\nu\) satisfies the two conditions.

\textcolor{blue}{(1): \(\nu(\emptyset) = 0\)}

Since both \(P\) and \(Q\) are finite measures, they satisfy \(P(\emptyset) = 0\) and \(Q(\emptyset) = 0\). Thus,
\[ \nu(\emptyset) = Q(\emptyset) - P(\emptyset) = 0 - 0 = 0. \]
Hence, \(\nu(\emptyset) = 0\), satisfying condition (1).\\
\textcolor{blue}{(2):\(\nu\) is \(\sigma\)-additive.}
To show that \(\nu\) is \(\sigma\)-additive, we need to demonstrate that for any countable collection of disjoint sets \(\{A_n\}\) in \(\mathcal{A}\),
\[ \nu\left(\bigcup_{n=1}^\infty A_n\right) = \sum_{n=1}^\infty \nu(A_n). \]

Since \(P\) and \(Q\) are finite measures, they are \(\sigma\)-additive. Therefore, we have
\[ P\left(\bigcup_{n=1}^\infty A_n\right) = \sum_{n=1}^\infty P(A_n) \]
and
\[ Q\left(\bigcup_{n=1}^\infty A_n\right) = \sum_{n=1}^\infty Q(A_n). \]

Now, consider \(\nu\) applied to the union of the disjoint sets \(\{A_n\}\):
\[
\nu\left(\bigcup_{n=1}^\infty A_n\right) = Q\left(\bigcup_{n=1}^\infty A_n\right) - P\left(\bigcup_{n=1}^\infty A_n\right).
\]

Using the \(\sigma\)-additivity of \(Q\) and \(P\), this becomes
\[
\nu\left(\bigcup_{n=1}^\infty A_n\right) = \sum_{n=1}^\infty Q(A_n) - \sum_{n=1}^\infty P(A_n).
\]

Since \(\nu(A_n) = Q(A_n) - P(A_n)\) for each \(n\), we have
\[
\sum_{n=1}^\infty \nu(A_n) = \sum_{n=1}^\infty (Q(A_n) - P(A_n)).
\]

Thus,
\[
\nu\left(\bigcup_{n=1}^\infty A_n\right) = \sum_{n=1}^\infty Q(A_n) - \sum_{n=1}^\infty P(A_n) = \sum_{n=1}^\infty (Q(A_n) - P(A_n)) = \sum_{n=1}^\infty \nu(A_n).
\]

This shows that \(\nu\) is \(\sigma\)-additive, satisfying condition (2).

We have verified that \(\nu := Q - P\) satisfies both conditions confirming that \(\nu\) is indeed a finite signed measure.
\end{myproof}
\ex{Finite signed measure $2$}{
\begin{align}
    \nu(A):=\int_A f\,d\mu=\int\mathbf{1}_Af\,d\mu
\end{align}
with a measure $\mu$ on $(\Omega,\SA)$ and a function $f\in\mathcal{L}^1(\mu)$.i.e.,$f:\Omega\to\bar{\mathbb{R}}$ is $\SA-$measurable with $\int|f|\,d\mu<\infty$.
}
\begin{myproof}
    \textcolor{red}{Main idea: By means of linearity of integrals and dominated convergence.}\\
Given:
\begin{itemize}
    \item A measure \(\mu\) on \((\Omega, \mathcal{A})\).
    \item A function \(f \in \mathcal{L}^1(\mu)\), i.e., \(f: \Omega \to \bar{\mathbb{R}}\) is \(\mathcal{A}\)-measurable with \(\int |f| \, d\mu < \infty\).
\end{itemize}

\textcolor{blue}{Check Condition (1):\(\nu(\emptyset) = 0\).}

By definition,
\[ \nu(\emptyset) = \int_\emptyset f \, d\mu. \]

Since the integral over the empty set is zero,
\[ \nu(\emptyset) = 0. \]

\textcolor{blue}{Check Condition (2): \(\nu\) is \(\sigma\)-additive.}

By definition,
\[ \nu\left(\bigcup_{n=1}^\infty A_n\right) = \int_{\bigcup_{n=1}^\infty A_n} f \, d\mu. \]

Since the sets \(A_n\) are disjoint, we can apply the linearity and \(\sigma\)-additivity of the integral:

\[ \int_{\bigcup_{n=1}^\infty A_n} f \, d\mu = \int \mathbf{1}_{\bigcup_{n=1}^\infty A_n} f \, d\mu = \int \left( \sum_{n=1}^\infty \mathbf{1}_{A_n} \right) f \, d\mu. \]

By the Monotone Convergence Theorem (or Dominated Convergence Theorem), we have:

\[ \int \left( \sum_{n=1}^\infty \mathbf{1}_{A_n} \right) f \, d\mu = \sum_{n=1}^\infty \int \mathbf{1}_{A_n} f \, d\mu = \sum_{n=1}^\infty \nu(A_n). \]

Thus,
\[ \nu\left(\bigcup_{n=1}^\infty A_n\right) = \sum_{n=1}^\infty \nu(A_n). \]

This shows that \(\nu\) is \(\sigma\)-additive, satisfying condition (2).

Since \(\nu\) satisfies both (1) and (2), we have shown that \(\nu(A) := \int_A f \, d\mu\) defines a finite signed measure on \((\Omega, \mathcal{A})\).

\[
\boxed{\nu(A) = \int_A f \, d\mu \text{ is a finite signed measure.}}
\]
\end{myproof}
\nt{Additivity: A finite signed measure $\nu$ is additive in the sense that
\begin{align}
    \nu(\bigcup^N_{n=1}A_n)=\sum_{n=1}^N\nu(A_n)
\end{align}for arbitrary $N\in\mathbb{N}$ and $A_1,A_2,\cdots,A_N\in\SA$.}
\ex{Continuity properties of signed measures}{
Let $\nu$ be a finite signed measure on $(\Omega,\SA)$. Show that for arbitary sets $B_1\subset B_2\subset B_3\subset\cdots$ in $\SA$,
\begin{align*}
    \nu(\bigcup_{n=1}^{\infty}B_n)=\lim_{n\to\infty}\nu(B_n_
    ).
\end{align*}
Show that for arbitrary sets $C_1\supset C_2\supset C_3\supset\cdots$ in $\SA$,
\begin{align*}
    \nu(\bigcap_{n=1}^{\infty}C_n)=\lim_{n\to\infty}\nu(C_n).
\end{align*}
}
\begin{myproof}
    Let's prove the two statements given for a finite signed measure \(\nu\) on \((\Omega, \mathcal{A})\).

\textcolor{blue}{1. Monotone Convergence for Increasing Sequence of Sets}

Statement: For arbitrary sets \(B_1 \subset B_2 \subset B_3 \subset \cdots\) in \(\mathcal{A}\),
\[ \nu\left(\bigcup_{n=1}^\infty B_n\right) = \lim_{n \to \infty} \nu(B_n). \]


Define \(B = \bigcup_{n=1}^\infty B_n\). The sequence \(\{B_n\}\) is increasing, so \(B_n \subset B_{n+1}\) for all \(n\). 

Using the definition of \(\nu\) as a signed measure, we can express \(\nu(B)\) as the limit of \(\nu(B_n)\).

First, observe that:
\[ B = \bigcup_{n=1}^\infty B_n. \]

Since the \(B_n\) are increasing,
\[ \nu(B_n) \leq \nu(B_{n+1}) \]
for all \(n\). Thus, the sequence \(\{\nu(B_n)\}\) is monotone increasing.

Because \(\nu\) is a finite signed measure, it is bounded. Hence, the monotone sequence \(\{\nu(B_n)\}\) converges to its supremum:
\[ \lim_{n \to \infty} \nu(B_n) = \sup_n \nu(B_n). \]

Consider the \(\sigma\)-additivity of \(\nu\),we have shown that:
\[ \nu\left(\bigcup_{n=1}^\infty B_n\right) = \lim_{n \to \infty} \nu(B_n). \]

\textcolor{blue}{2. Monotone Convergence for Decreasing Sequence of Sets}

Statement: For arbitrary sets \(C_1 \supset C_2 \supset C_3 \supset \cdots\) in \(\mathcal{A}\),
\[ \nu\left(\bigcap_{n=1}^\infty C_n\right) = \lim_{n \to \infty} \nu(C_n). \]


Define \(C = \bigcap_{n=1}^\infty C_n\). The sequence \(\{C_n\}\) is decreasing, so \(C_{n+1} \subset C_n\) for all \(n\).

Using the definition of \(\nu\) as a signed measure, we can express \(\nu(C)\) as the limit of \(\nu(C_n)\).

First, observe that:
\[ C = \bigcap_{n=1}^\infty C_n. \]

Since the \(C_n\) are decreasing,
\[ \nu(C_{n+1}) \leq \nu(C_n) \]
for all \(n\). Thus, the sequence \(\{\nu(C_n)\}\) is monotone decreasing.

Because \(\nu\) is a finite signed measure, it is bounded. Hence, the monotone sequence \(\{\nu(C_n)\}\) converges to its infimum:
\[ \lim_{n \to \infty} \nu(C_n) = \inf_n \nu(C_n). \]

Consider the \(\sigma\)-additivity of \(\nu\),we have shown that:
\[ \nu\left(\bigcap_{n=1}^\infty C_n\right) = \lim_{n \to \infty} \nu(C_n). \]

We have proven both statements for a finite signed measure \(\nu\) on \((\Omega, \mathcal{A})\):

1. For an increasing sequence of sets \(\{B_n\}\):
\[ \nu\left(\bigcup_{n=1}^\infty B_n\right) = \lim_{n \to \infty} \nu(B_n). \]

2. For a decreasing sequence of sets \(\{C_n\}\):
\[ \nu\left(\bigcap_{n=1}^\infty C_n\right) = \lim_{n \to \infty} \nu(C_n). \]
\end{myproof}
\dfn{Positive and negative sets}{
Let $\nu$ be a finite signed measure on $(\Omega,\SA)$. A set $A_{\star}\subset\Omega$ is called $\nu-$positive if $A_{\star}\in\SA$ and 
\begin{align}
    \nu(A)\geq 0 \text{ for all }A\in\SA \text{ with }A\subset A_{\star}.
\end{align}
 A set $A_{\star}\subset\Omega$ is called $\nu-$negative if $A_{\star}\in\SA$ and 
\begin{align}
    \nu(A)\leq 0 \text{ for all }A\in\SA \text{ with }A\subset A_{\star}.
\end{align}
}
\mprop{Existence of nontrivial positive sets}{
Let $\nu$ be a finite signed measure on $(\Omega,\SA)$, and let $A_0\in\SA$ with $\nu(A_0)>0$. Then there exists a $\nu-$positive set $A_{\star}\subset A_0$ with  $\nu(A_{\star})\geq\nu(A_0)$.
}
\begin{myproof}
    Define
    \begin{align*}
        \delta_0:=\sup\{-\nu(B):B\in\SA,B\subset A_0\}\geq 0.
    \end{align*}Write $A_0=A_1\cup B_1$ with \textcolor{blue}{disjoint measurable sets $A_1$ and $B_1$} such that
    \begin{align*}
        -\nu(B_1)\geq\min\{\frac{\delta_0}{2},1\}.
    \end{align*}[Iterated]:After $k$ steps we have measurable sets $A_0\supset A_1\supset\cdots\supset A_k$, and we consider the number 
    \begin{align*}
        \delta_k:=\sup\{-\nu(B):B\in\SA,B\subset A_k\}\geq 0.
    \end{align*}Then we write $A_k=A_{k+1}\cup B_{k+1}$ with disjoint measurable sets $A_{k+1}$ and $B_{k+1}$ such that 
    \begin{align*}
        -\nu(B_{k+1})\geq\min\{\frac{\delta_k}{2},1\}.
    \end{align*}This construction yields a non-increasing sequence $(A_k)_{k\geq 0}$ and the disjoint sets $B_k=A_{k-1}\backslash A_k,k\geq 1$. We may write 
    \begin{align*}
        A_0=A_{\star}\cup B_{\star}
    \end{align*}with the disjoint sets
    \begin{align*}
        A_{\star}=\cap_{k\geq 0}A_k\text{ and }B_{\star}=\cup_{k\geq 1}B_k.
    \end{align*}Since $\nu(B_{\star})=\sum_{k=1}^{\infty}\nu(B_k)$ with nonpositive summands $\nu(B_k)$, we obtain the inequality
    \begin{align*}
        \nu(A_{\star})=\nu(A_0)-\nu(B_{\star})\geq\nu(A_0).
    \end{align*}Moreover, the sequence $(\nu(B_k))_{k\geq 1}$ converges to $0$, so the inequalities 
    \begin{align*}
       0\leq \min\{\frac{\delta_k}{2},1\}\leq -\nu(B_{k+1})\Rightarrow \lim_{k\to\infty}\delta_k=0.
    \end{align*}Consequently, for any measurable set $A\subset A_{\star}$,
    \begin{align*}
        -\nu(A)\leq\inf_{k\geq 0}\sup\{-\nu(A'):A'\in\SA,A'\subset A_k\}=\inf_{k\geq 0}\delta_k=0,
    \end{align*}whence $\nu(A)\geq0$. This shows that $A_{\star}$ is a $\nu-$positive set.
\end{myproof}
\ex{Unions of positive sets}{
Let $\nu$ be a finite signed measure on $(\Omega,\SA)$. Let $(A_n)_{n\geq 1}$ be a sequence of $\nu-$positive sets. Show that
}
\section{Radon-Nikodym Derivatives}

\subsection{Finite measures}


\subsection{Absolute continuity and $\sigma-$finite measures}


\chapter{Abstract Integrals}
\section{Lattices and Stone Lattices}

\section{Abstract and Usual Integrals}

\section{Representations of Dual Spaces}

\section{Prohorov's Theorem}

\end{document}
