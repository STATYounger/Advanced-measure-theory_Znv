\documentclass{report}

\input{preamble}
\input{macros}
\input{letterfonts}

\title{\Huge{Advanced Measure Theory}}
\author{Based by Prof.Dr.Dümbgen Lutz\\ZnV\\
\textit{The life is a function of the value you bring to others.}}
\date{\today}

\begin{document}

\maketitle
\newpage% or \cleardoublepage
% \pdfbookmark[<level>]{<title>}{<dest>}
\pdfbookmark[section]{\contentsname}{toc}
\tableofcontents
\pagebreak

\chapter{Signed Measures}
\textit{Start the coming trip from \textcolor{blue}{finite signed measures}.}\\
Assume $(\Omega,\SA)$ be a measurable space, where $\Omega$ is a nonempty set equipped with a $\sigma-$algebra $\SA$ over $\Omega$.
\section{The Hahn-Jordan Decomposition}
\dfn{Finite signed measure}{
A function $\nu:\SA\to\mathbb{R}$ is called a finite signed measure on $(\Omega,\SA)$, if
\begin{enumerate}
    \item $\nu(\emptyset)=0$,
    \item $\nu$ is $\sigma-$additive.
\end{enumerate}
}
\nt{$\sigma-$additive: for arbitrary disjoint sets $A_1,A_2,A_3,\cdots\in\SA$,
\begin{align*}
    \nu(\bigcup_{n=1}^{\infty}A_n)=\sum_{n=1}^{\infty}\nu(A_n).
\end{align*}}
\qs{}{What is the difference to a finite measure?}
$\nu(A)$ may be negative for some sets $A\in\SA$.
\qs{}{Why has the series $\sum_{n=1}^{\infty}\nu(A_n)$ converge absolutely?}
Because the set $\bigcup_{n=1}A_n$ is invariant w.r.t.reorderings of the sets $A_1,A_2,\cdots$.

\ex{Finite signed measure $1$}{
\begin{align}
    \nu:=Q-P
\end{align}with finite measures $P,Q$ on $(\Omega,\SA)$.
}
\begin{myproof}
To prove that \(\nu := Q - P\) with finite measures \(P\) and \(Q\) on \((\Omega, \mathcal{A})\) is a finite signed measure, we need to verify that \(\nu\) satisfies the two conditions.

\textcolor{blue}{(1): \(\nu(\emptyset) = 0\)}

Since both \(P\) and \(Q\) are finite measures, they satisfy \(P(\emptyset) = 0\) and \(Q(\emptyset) = 0\). Thus,
\[ \nu(\emptyset) = Q(\emptyset) - P(\emptyset) = 0 - 0 = 0. \]
Hence, \(\nu(\emptyset) = 0\), satisfying condition (1).\\
\textcolor{blue}{(2):\(\nu\) is \(\sigma\)-additive.}
To show that \(\nu\) is \(\sigma\)-additive, we need to demonstrate that for any countable collection of disjoint sets \(\{A_n\}\) in \(\mathcal{A}\),
\[ \nu\left(\bigcup_{n=1}^\infty A_n\right) = \sum_{n=1}^\infty \nu(A_n). \]

Since \(P\) and \(Q\) are finite measures, they are \(\sigma\)-additive. Therefore, we have
\[ P\left(\bigcup_{n=1}^\infty A_n\right) = \sum_{n=1}^\infty P(A_n) \]
and
\[ Q\left(\bigcup_{n=1}^\infty A_n\right) = \sum_{n=1}^\infty Q(A_n). \]

Now, consider \(\nu\) applied to the union of the disjoint sets \(\{A_n\}\):
\[
\nu\left(\bigcup_{n=1}^\infty A_n\right) = Q\left(\bigcup_{n=1}^\infty A_n\right) - P\left(\bigcup_{n=1}^\infty A_n\right).
\]

Using the \(\sigma\)-additivity of \(Q\) and \(P\), this becomes
\[
\nu\left(\bigcup_{n=1}^\infty A_n\right) = \sum_{n=1}^\infty Q(A_n) - \sum_{n=1}^\infty P(A_n).
\]

Since \(\nu(A_n) = Q(A_n) - P(A_n)\) for each \(n\), we have
\[
\sum_{n=1}^\infty \nu(A_n) = \sum_{n=1}^\infty (Q(A_n) - P(A_n)).
\]

Thus,
\[
\nu\left(\bigcup_{n=1}^\infty A_n\right) = \sum_{n=1}^\infty Q(A_n) - \sum_{n=1}^\infty P(A_n) = \sum_{n=1}^\infty (Q(A_n) - P(A_n)) = \sum_{n=1}^\infty \nu(A_n).
\]

This shows that \(\nu\) is \(\sigma\)-additive, satisfying condition (2).

We have verified that \(\nu := Q - P\) satisfies both conditions confirming that \(\nu\) is indeed a finite signed measure.
\end{myproof}
\ex{Finite signed measure $2$}{
\begin{align}
    \nu(A):=\int_A f\,d\mu=\int\mathbf{1}_Af\,d\mu
\end{align}
with a measure $\mu$ on $(\Omega,\SA)$ and a function $f\in\mathcal{L}^1(\mu)$.i.e.,$f:\Omega\to\bar{\mathbb{R}}$ is $\SA-$measurable with $\int|f|\,d\mu<\infty$.
}
\begin{myproof}
    \textcolor{red}{Main idea: By means of linearity of integrals and dominated convergence.}\\
Given:
\begin{itemize}
    \item A measure \(\mu\) on \((\Omega, \mathcal{A})\).
    \item A function \(f \in \mathcal{L}^1(\mu)\), i.e., \(f: \Omega \to \bar{\mathbb{R}}\) is \(\mathcal{A}\)-measurable with \(\int |f| \, d\mu < \infty\).
\end{itemize}

\textcolor{blue}{Check Condition (1):\(\nu(\emptyset) = 0\).}

By definition,
\[ \nu(\emptyset) = \int_\emptyset f \, d\mu. \]

Since the integral over the empty set is zero,
\[ \nu(\emptyset) = 0. \]

\textcolor{blue}{Check Condition (2): \(\nu\) is \(\sigma\)-additive.}

By definition,
\[ \nu\left(\bigcup_{n=1}^\infty A_n\right) = \int_{\bigcup_{n=1}^\infty A_n} f \, d\mu. \]

Since the sets \(A_n\) are disjoint, we can apply the linearity and \(\sigma\)-additivity of the integral:

\[ \int_{\bigcup_{n=1}^\infty A_n} f \, d\mu = \int \mathbf{1}_{\bigcup_{n=1}^\infty A_n} f \, d\mu = \int \left( \sum_{n=1}^\infty \mathbf{1}_{A_n} \right) f \, d\mu. \]

By the Monotone Convergence Theorem (or Dominated Convergence Theorem), we have:

\[ \int \left( \sum_{n=1}^\infty \mathbf{1}_{A_n} \right) f \, d\mu = \sum_{n=1}^\infty \int \mathbf{1}_{A_n} f \, d\mu = \sum_{n=1}^\infty \nu(A_n). \]

Thus,
\[ \nu\left(\bigcup_{n=1}^\infty A_n\right) = \sum_{n=1}^\infty \nu(A_n). \]

This shows that \(\nu\) is \(\sigma\)-additive, satisfying condition (2).

Since \(\nu\) satisfies both (1) and (2), we have shown that \(\nu(A) := \int_A f \, d\mu\) defines a finite signed measure on \((\Omega, \mathcal{A})\).

\[
\boxed{\nu(A) = \int_A f \, d\mu \text{ is a finite signed measure.}}
\]
\end{myproof}

\section{Radon-Nikodym Derivatives}

\subsection{Finite measures}


\subsection{Absolute continuity and $\sigma-$finite measures}


\chapter{Abstract Integrals}
\section{Lattices and Stone Lattices}

\section{Abstract and Usual Integrals}

\section{Representations of Dual Spaces}

\section{Prohorov's Theorem}

\end{document}
